\documentclass{article}
\usepackage[spanish]{babel}
\usepackage{amsthm}
\usepackage{amsmath, amssymb}
\usepackage{lmodern}

\theoremstyle{plain}
\newtheorem{theorem}{Teorema}[section]
\newtheorem{corollary}{Corolario}[theorem]
\newtheorem{lemma}[theorem]{Lema}
\renewcommand\qedsymbol{$\blacksquare$}

\begin{document}

\section{Teorema Fundamental de la Aritmética}

\begin{theorem}[Teorema Fundamental de la Aritmética]
Todo entero mayor que 1 puede escribirse de forma única como un producto de números primos, salvo el orden de los factores.
\end{theorem}

\begin{proof}
La demostración se divide en dos partes: existencia y unicidad.

\textbf{Existencia:} Procedemos por inducción sobre \(n \in \mathbb{N}\), con \(n > 1\).

Caso base: \(n = 2\). Como 2 es primo, ya es producto de un único primo.

Paso inductivo: Supongamos que todo entero \(k\), con \(2 \leq k < n\), se puede expresar como producto de primos. Si \(n\) es primo, ya está expresado como producto de primos. Si \(n\) no es primo, entonces existe \(a, b\) tales que \(n = ab\), con \(1 < a < n\) y \(1 < b < n\). Por hipótesis inductiva, \(a\) y \(b\) se escriben como productos de primos, por lo tanto \(n\) también.

\textbf{Unicidad:} Supongamos que un número \(n\) tiene dos descomposiciones distintas en primos:
\[
n = p_1 p_2 \cdots p_r = q_1 q_2 \cdots q_s,
\]
donde todos los \(p_i\) y \(q_j\) son primos. Usamos inducción y el hecho de que si un primo \(p\) divide un producto, entonces divide al menos uno de los factores (propiedad fundamental de los primos).

Se puede mostrar que \(p_1\) debe coincidir con alguno de los \(q_j\). Reordenando, cancelamos ese factor común y repetimos el argumento. Finalmente, llegamos a que ambas factorizaciones son iguales salvo el orden.

\end{proof}

\begin{corollary}
La cantidad de representaciones de un número natural como producto de primos es finita y única, salvo el orden de los factores.
\end{corollary}

\begin{lemma}
Si un número primo \(p\) divide al producto \(ab\), entonces \(p\) divide a \(a\) o a \(b\).
\end{lemma}

\end{document}